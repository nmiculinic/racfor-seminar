\documentclass[times, utf8, seminar, english]{fer}
\usepackage{booktabs}
\usepackage{epigraph}
\usepackage{mathtools, amsmath,amsfonts,amssymb, amsthm}
\usepackage{hyperref}
\usepackage{fancybox}
\usepackage{graphicx}
\usepackage{etoolbox}
\makeatletter
\patchcmd{\chapter}{\if@openright\cleardoublepage\else\clearpage\fi}{}{}{}
\makeatother

\begin{document}
\theoremstyle{definition}
\newtheorem{definition}{Definition}[section]

\title{iproute2 and iptables packet}
\author{Neven Miculinić}

\maketitle
\tableofcontents

\chapter{Introduction}

Networking is one of the most important topic in everyday computer use. Every day you send and recieve thousands IP packages.
Every facebook like, youtube video view or application download utilized world wide web, that is the Internet.

Therefore, it's more than likely you'd come into contact with linux networking stack and tooling.
Since this topic is masively broad and complex, this paper is focusing on following two components:
\begin{itemize}
    \item Netfilter/iptables --- framework for various network related operations for packet filtering, NAT and port translation
    \item iproute2 --- usespace utilies for controling and monitoring various networking options in Linux kernel
\end{itemize}

The essay is structued with brief introduction in packets flow through various Linux kernel utilities. Figure~\ref


\begin{figure}
  \caption{A picture of a gull.}
  \centering
    \includegraphics[]{table_traverse}
\end{figure}

\section{iptables}
\ovalbox{
    \begin{minipage}{.8\textwidth}
        \textit{Netfilter is a framework provided by Linux that allows various networking-related operations to be implemented in the form of customized handlers. Netfilter offers various functions and operations for packet filtering, network address translation, and port translation, which provide the functionality required for directing packets through a network, as well as for providing ability to prohibit packets from reaching sensitive locations within a computer network.
        }
        \par\emph{Wikipedia}
    \end{minipage}
}
\\

nftables is newer framework, intended to replace Netfilter; however this section focuses on older Netfilter and its usages due to wide support.
This section focuses on the iptables, one of the utilites provided by Netfilter framework for managing IP packages in the Linux kernel.

It's userspace utility for configuring Linux kernel firewall within Netfilter project.:wq



\section{Routing tables}
\section{Network namespaces}


\chapter{Example usecases}
pass
\section{OpenVPN on Google cloud platform}
pass
\section{Isolating process in its own network namespace}

\bibliography{refs}
\bibliographystyle{fer}

\end{document}
