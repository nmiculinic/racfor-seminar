\documentclass[times, utf8, seminar, english]{fer}
\usepackage{booktabs}
\usepackage{epigraph}
\usepackage{mathtools, amsmath,amsfonts,amssymb, amsthm}
\usepackage{hyperref}

\usepackage{etoolbox}
\makeatletter
\patchcmd{\chapter}{\if@openright\cleardoublepage\else\clearpage\fi}{}{}{}
\makeatother

\begin{document}
\theoremstyle{definition}
\newtheorem{definition}{Definition}[section]

\title{iproute2 and iptables packet}
\author{Neven Miculinić}

\maketitle
\tableofcontents

\chapter{Introduction}

Networking is one of the most important topic in everyday computer use. Every day you send and recieve thousand IP packages. 
Every facebook like, youtube video view or application download utilized world wide web, that is the Internet.
Overwhelming majority of the Internet infrastucture Linux, 98.3\%...check data

Therefore, it's more than likely you'd come into contact with linux networking stack and tooling.
Since this topic is masively broad and complex, this paper is focusing on following two components:
\begin{itemize}
    \item Netfilter/iptables --- framework for various network related operations for packet filtering, NAT and port translation
    \item iproute2 --- usespace utilies for controling and monitoring various networking options in Linux kernel
\end{itemize}

\chapter{Brief overview of Linux kernel network stack}
pass
\chapter{Example usecases}
pass
\section{OpenVPN on Google cloud platform}
pass
\section{Isolating process in its own network namespace}

\bibliography{refs}
\bibliographystyle{fer}

\end{document}
